% This file is part of The Thresher.

% to-do
% -----
% - figure out what to do
% - do it
% - write first draft
% - make and insert figures

\documentclass[12pt,preprint]{aastex}

\newcounter{address}
\setcounter{address}{0}
\newcommand{\foreign}[1]{\textit{#1}}
\newcommand{\etal}{\foreign{et~al.}}
\newcommand{\project}[1]{\textsl{#1}}
\newcommand{\TheThresher}{\project{The~Thresher}}
\newcommand{\documentname}{\textsl{Article}}

\newcommand{\given}{\,|\,}
\newcommand{\transpose}[1]{{#1}^{\mathsf{T}}}

\title{Non-negative priors or regularizations: Costs and benefits}
\author{
  David~W.~Hogg\altaffilmark{\ref{CCPP},\ref{MPIA},\ref{email}}
  and others
}

\setcounter{address}{1}
\altaffiltext{\theaddress}{\stepcounter{address}\label{CCPP} Center
  for Cosmology and Particle Physics, Department of Physics, New York
  University, 4 Washington Place, New York, NY 10003}
\altaffiltext{\theaddress}{\stepcounter{address}\label{MPIA}
  Max-Planck-Institut f\"ur Astronomie, K\"onigstuhl 17, D-69117
  Heidelberg, Germany}
\altaffiltext{\theaddress}{\stepcounter{address}\label{email} To whom
  correspondence should be addressed: \texttt{david.hogg@nyu.edu}}

\begin{abstract}
Many processes in astronomy can be thought of as non-negative
superpositions of non-negative components (think, for example, of
stars and sky foreground summing to make a telescope image of a
galaxy).  Inference of astrophysical phenomena can often therefore be
performed with non-negative priors or priors that rule out all
negative numbers in superpositions.  In, for example, a data-driven
model of a spectrum with 2000 pixels, the non-negative prior cuts out
all but $2^{-2000}$ of the \foreign{a priori} parameter space, so it
is extremely informative and helpful for inference.  At the same time,
non-negative priors ensure that uncertainties or noise---in the inputs
or the outputs of the inference---cannot be Gaussian, so they
explicitly invalidate chi-squared fitting or any other simple Gaussian
approaches, including all the standard methods by which photometry
(brightness measurement) is performed in imaging.  We give simple
examples in which non-negative priors are clearly justified but lead
to strongly biased point estimates.  These problems can all be
overcome in fully probabilistic approaches to data analysis, but these
involve passing forward probability distributions for everything in
every stage of data analysis.  There are currently no practical
proposals for performing these kinds of analyses for typical
astrophysical data sets.
\end{abstract}

\begin{document}

\noindent\textbf{[This document is a very early draft
    dated 2012-05-25.  Please do not cite it
    without the permission of the authors.]}

Fake data:

A set of N noisy images with different sky levels and PSFs.  Each
image has two stars, one bright and one faint.

Experiments:

Estimate sky + scene for each image, with and without non-negative.
Biases?

Estimate PSF in each image, with and without non-negative.  Biases?

Find scene + sky levels that explain all N images, with and without
non-negative, with and without band limited.  Do with the correct true
PSF.

Do again with the estimated NN PSF.  Even more bias?

Repeat the scene and sky estimation and PSF estimation with posterior
PDF sampling.  What problems does this solve?  What problems does this
introduce?

\acknowledgements It is a pleasure to thank
  Mike Blanton (NYU),
  Adam Bolton (Utah),
  Brendon Brewer (UCSB),
  Dan Foreman-Mackey (NYU),
  Stefan Harmeling (T\"ubingen),
  Michael Hirsch (UCL),
  Dustin Lang (CMU),
  Phil Marshall (Oxford), and
  Bernhard Sch\"olkopf (T\"ubingen)
for contributions and comments.  DWH was partially
supported by NASA (grant NNX12AI50G) and the NSF (grant
IIS-1124794).  This project made use of the NASA \project{Astrophysics
  Data System}, and code in the the \project{numpy}, \project{scipy},
and \project{matplotlib} open-source projects.  All the code used in
this paper is available at [URL HERE].

\begin{thebibliography}{70}
\bibitem[Hirsch \etal(2011)]{hirsch}
Hirsch,~M., Harmeling,~S., Sra,~S., Sch\"olkopf,~B., 2011, \aap, 531, A9
\bibitem[Magain \etal(1998)]{magain}
Magain,~P., Courbin,~F., Sohy,~S., 1998, \apj, 494, 472 
\end{thebibliography}

\end{document}
